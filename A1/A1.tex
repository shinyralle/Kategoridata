\documentclass[]{article}
\usepackage{lmodern}
\usepackage{amssymb,amsmath}
\usepackage{ifxetex,ifluatex}
\usepackage{fixltx2e} % provides \textsubscript
\ifnum 0\ifxetex 1\fi\ifluatex 1\fi=0 % if pdftex
  \usepackage[T1]{fontenc}
  \usepackage[utf8]{inputenc}
\else % if luatex or xelatex
  \ifxetex
    \usepackage{mathspec}
  \else
    \usepackage{fontspec}
  \fi
  \defaultfontfeatures{Ligatures=TeX,Scale=MatchLowercase}
\fi
% use upquote if available, for straight quotes in verbatim environments
\IfFileExists{upquote.sty}{\usepackage{upquote}}{}
% use microtype if available
\IfFileExists{microtype.sty}{%
\usepackage{microtype}
\UseMicrotypeSet[protrusion]{basicmath} % disable protrusion for tt fonts
}{}
\usepackage[margin=1in]{geometry}
\usepackage{hyperref}
\hypersetup{unicode=true,
            pdftitle={A1},
            pdfauthor={Magnus Brogie, Rahul Duggal},
            pdfborder={0 0 0},
            breaklinks=true}
\urlstyle{same}  % don't use monospace font for urls
\usepackage{color}
\usepackage{fancyvrb}
\newcommand{\VerbBar}{|}
\newcommand{\VERB}{\Verb[commandchars=\\\{\}]}
\DefineVerbatimEnvironment{Highlighting}{Verbatim}{commandchars=\\\{\}}
% Add ',fontsize=\small' for more characters per line
\usepackage{framed}
\definecolor{shadecolor}{RGB}{248,248,248}
\newenvironment{Shaded}{\begin{snugshade}}{\end{snugshade}}
\newcommand{\AlertTok}[1]{\textcolor[rgb]{0.94,0.16,0.16}{#1}}
\newcommand{\AnnotationTok}[1]{\textcolor[rgb]{0.56,0.35,0.01}{\textbf{\textit{#1}}}}
\newcommand{\AttributeTok}[1]{\textcolor[rgb]{0.77,0.63,0.00}{#1}}
\newcommand{\BaseNTok}[1]{\textcolor[rgb]{0.00,0.00,0.81}{#1}}
\newcommand{\BuiltInTok}[1]{#1}
\newcommand{\CharTok}[1]{\textcolor[rgb]{0.31,0.60,0.02}{#1}}
\newcommand{\CommentTok}[1]{\textcolor[rgb]{0.56,0.35,0.01}{\textit{#1}}}
\newcommand{\CommentVarTok}[1]{\textcolor[rgb]{0.56,0.35,0.01}{\textbf{\textit{#1}}}}
\newcommand{\ConstantTok}[1]{\textcolor[rgb]{0.00,0.00,0.00}{#1}}
\newcommand{\ControlFlowTok}[1]{\textcolor[rgb]{0.13,0.29,0.53}{\textbf{#1}}}
\newcommand{\DataTypeTok}[1]{\textcolor[rgb]{0.13,0.29,0.53}{#1}}
\newcommand{\DecValTok}[1]{\textcolor[rgb]{0.00,0.00,0.81}{#1}}
\newcommand{\DocumentationTok}[1]{\textcolor[rgb]{0.56,0.35,0.01}{\textbf{\textit{#1}}}}
\newcommand{\ErrorTok}[1]{\textcolor[rgb]{0.64,0.00,0.00}{\textbf{#1}}}
\newcommand{\ExtensionTok}[1]{#1}
\newcommand{\FloatTok}[1]{\textcolor[rgb]{0.00,0.00,0.81}{#1}}
\newcommand{\FunctionTok}[1]{\textcolor[rgb]{0.00,0.00,0.00}{#1}}
\newcommand{\ImportTok}[1]{#1}
\newcommand{\InformationTok}[1]{\textcolor[rgb]{0.56,0.35,0.01}{\textbf{\textit{#1}}}}
\newcommand{\KeywordTok}[1]{\textcolor[rgb]{0.13,0.29,0.53}{\textbf{#1}}}
\newcommand{\NormalTok}[1]{#1}
\newcommand{\OperatorTok}[1]{\textcolor[rgb]{0.81,0.36,0.00}{\textbf{#1}}}
\newcommand{\OtherTok}[1]{\textcolor[rgb]{0.56,0.35,0.01}{#1}}
\newcommand{\PreprocessorTok}[1]{\textcolor[rgb]{0.56,0.35,0.01}{\textit{#1}}}
\newcommand{\RegionMarkerTok}[1]{#1}
\newcommand{\SpecialCharTok}[1]{\textcolor[rgb]{0.00,0.00,0.00}{#1}}
\newcommand{\SpecialStringTok}[1]{\textcolor[rgb]{0.31,0.60,0.02}{#1}}
\newcommand{\StringTok}[1]{\textcolor[rgb]{0.31,0.60,0.02}{#1}}
\newcommand{\VariableTok}[1]{\textcolor[rgb]{0.00,0.00,0.00}{#1}}
\newcommand{\VerbatimStringTok}[1]{\textcolor[rgb]{0.31,0.60,0.02}{#1}}
\newcommand{\WarningTok}[1]{\textcolor[rgb]{0.56,0.35,0.01}{\textbf{\textit{#1}}}}
\usepackage{longtable,booktabs}
\usepackage{graphicx,grffile}
\makeatletter
\def\maxwidth{\ifdim\Gin@nat@width>\linewidth\linewidth\else\Gin@nat@width\fi}
\def\maxheight{\ifdim\Gin@nat@height>\textheight\textheight\else\Gin@nat@height\fi}
\makeatother
% Scale images if necessary, so that they will not overflow the page
% margins by default, and it is still possible to overwrite the defaults
% using explicit options in \includegraphics[width, height, ...]{}
\setkeys{Gin}{width=\maxwidth,height=\maxheight,keepaspectratio}
\IfFileExists{parskip.sty}{%
\usepackage{parskip}
}{% else
\setlength{\parindent}{0pt}
\setlength{\parskip}{6pt plus 2pt minus 1pt}
}
\setlength{\emergencystretch}{3em}  % prevent overfull lines
\providecommand{\tightlist}{%
  \setlength{\itemsep}{0pt}\setlength{\parskip}{0pt}}
\setcounter{secnumdepth}{0}
% Redefines (sub)paragraphs to behave more like sections
\ifx\paragraph\undefined\else
\let\oldparagraph\paragraph
\renewcommand{\paragraph}[1]{\oldparagraph{#1}\mbox{}}
\fi
\ifx\subparagraph\undefined\else
\let\oldsubparagraph\subparagraph
\renewcommand{\subparagraph}[1]{\oldsubparagraph{#1}\mbox{}}
\fi

%%% Use protect on footnotes to avoid problems with footnotes in titles
\let\rmarkdownfootnote\footnote%
\def\footnote{\protect\rmarkdownfootnote}

%%% Change title format to be more compact
\usepackage{titling}

% Create subtitle command for use in maketitle
\newcommand{\subtitle}[1]{
  \posttitle{
    \begin{center}\large#1\end{center}
    }
}

\setlength{\droptitle}{-2em}

  \title{A1}
    \pretitle{\vspace{\droptitle}\centering\huge}
  \posttitle{\par}
    \author{Magnus Brogie, Rahul Duggal}
    \preauthor{\centering\large\emph}
  \postauthor{\par}
      \predate{\centering\large\emph}
  \postdate{\par}
    \date{12 November 2018}


\begin{document}
\maketitle

\hypertarget{assignment-1}{%
\section{Assignment 1}\label{assignment-1}}

\hypertarget{hur-manga-percent-av-man-respektive-kvinnor-ar-for-och-emot-laglig-abort}{%
\subsection{1. Hur många percent av män respektive kvinnor är för och
emot laglig
abort?}\label{hur-manga-percent-av-man-respektive-kvinnor-ar-for-och-emot-laglig-abort}}

\begin{Shaded}
\begin{Highlighting}[]
\KeywordTok{library}\NormalTok{(tidyverse)}
\end{Highlighting}
\end{Shaded}

\begin{verbatim}
## -- Attaching packages ---------------------------------- tidyverse 1.2.1 --
\end{verbatim}

\begin{verbatim}
## v ggplot2 3.1.0     v purrr   0.2.5
## v tibble  1.4.2     v dplyr   0.7.7
## v tidyr   0.8.2     v stringr 1.3.1
## v readr   1.1.1     v forcats 0.3.0
\end{verbatim}

\begin{verbatim}
## -- Conflicts ------------------------------------- tidyverse_conflicts() --
## x dplyr::filter() masks stats::filter()
## x dplyr::lag()    masks stats::lag()
\end{verbatim}

\begin{Shaded}
\begin{Highlighting}[]
\NormalTok{tab1<-}\StringTok{ }\KeywordTok{as.table}\NormalTok{(}\KeywordTok{rbind}\NormalTok{(}\KeywordTok{c}\NormalTok{(}\DecValTok{309}\NormalTok{, }\DecValTok{191}\NormalTok{), }\KeywordTok{c}\NormalTok{(}\DecValTok{319}\NormalTok{, }\DecValTok{281}\NormalTok{)))}
\KeywordTok{dimnames}\NormalTok{(tab1) <-}\StringTok{ }\KeywordTok{list}\NormalTok{(}\DataTypeTok{gender =} \KeywordTok{c}\NormalTok{(}\StringTok{"women"}\NormalTok{, }\StringTok{"men"}\NormalTok{),}\DataTypeTok{opinion =} \KeywordTok{c}\NormalTok{(}\StringTok{"favor"}\NormalTok{,}\StringTok{"against"}\NormalTok{))}

\CommentTok{# install.packages("kableExtra")}
\NormalTok{cont_table =}\StringTok{ }\KeywordTok{matrix}\NormalTok{(}
  \KeywordTok{c}\NormalTok{(}\DecValTok{309}\NormalTok{, }\DecValTok{191}\NormalTok{, }\DecValTok{319}\NormalTok{, }\DecValTok{281}\NormalTok{), }\CommentTok{# the data elements }
  \DataTypeTok{nrow=}\DecValTok{2}\NormalTok{,              }\CommentTok{# number of rows }
  \DataTypeTok{ncol=}\DecValTok{2}\NormalTok{,              }\CommentTok{# number of columns }
  \DataTypeTok{byrow =} \OtherTok{TRUE}\NormalTok{)  }\CommentTok{# fill matrix by rows }
\KeywordTok{rownames}\NormalTok{(cont_table) <-}\StringTok{  }\KeywordTok{c}\NormalTok{(}\StringTok{"Women"}\NormalTok{, }\StringTok{"Men"}\NormalTok{)}
\KeywordTok{colnames}\NormalTok{(cont_table) <-}\StringTok{  }\KeywordTok{c}\NormalTok{(}\StringTok{'In Favor'}\NormalTok{,}\StringTok{'Against'}\NormalTok{)}
\NormalTok{cont_table }\OperatorTok\StringTok{ }\NormalTok{knitr}\OperatorTok{::}\KeywordTok{kable}\NormalTok{()}
\end{Highlighting}
\end{Shaded}

\begin{longtable}[]{@{}lrr@{}}
\toprule
& In Favor & Against\tabularnewline
\midrule
\endhead
Women & 309 & 191\tabularnewline
Men & 319 & 281\tabularnewline
\bottomrule
\end{longtable}

\begin{Shaded}
\begin{Highlighting}[]
\NormalTok{women_men =}\StringTok{ }\KeywordTok{rowSums}\NormalTok{(cont_table)}
\NormalTok{favor_against =}\StringTok{ }\KeywordTok{colSums}\NormalTok{(cont_table)}
\NormalTok{total =}\StringTok{ }\KeywordTok{sum}\NormalTok{(cont_table)}

\CommentTok{# Calculates percentages by matrix multiplication}
\NormalTok{percentages_matrix <-}\StringTok{ }\NormalTok{(}\KeywordTok{diag}\NormalTok{(}\DecValTok{1} \OperatorTok{/}\StringTok{ }\NormalTok{women_men) }\OperatorTok\StringTok{ }\NormalTok{cont_table) }\OperatorTok{*}\StringTok{ }\DecValTok{100}
\KeywordTok{colnames}\NormalTok{(percentages_matrix) <-}\StringTok{ }\KeywordTok{c}\NormalTok{(}\StringTok{"In Favor"}\NormalTok{, }\StringTok{"Against"}\NormalTok{)}
\KeywordTok{rownames}\NormalTok{(percentages_matrix) <-}\StringTok{ }\KeywordTok{c}\NormalTok{(}\StringTok{"Women"}\NormalTok{, }\StringTok{"Men"}\NormalTok{)}
\CommentTok{# percentages_matrix #  %>% knitr::kable()}

\NormalTok{expected_matrix <-}\StringTok{ }\NormalTok{total }\OperatorTok{*}\StringTok{ }\NormalTok{(}\KeywordTok{diag}\NormalTok{(}\DecValTok{1} \OperatorTok{/}\StringTok{ }\NormalTok{women_men)  }\OperatorTok\StringTok{ }\NormalTok{cont_table) }\OperatorTok{*}\StringTok{ }\NormalTok{(cont_table }\OperatorTok\StringTok{ }\KeywordTok{diag}\NormalTok{(}\DecValTok{1} \OperatorTok{/}\StringTok{ }\NormalTok{favor_against))}
\CommentTok{# expected_matrix}
\CommentTok{# cont_table}
\NormalTok{cont_minus_expected <-}\StringTok{ }\NormalTok{cont_table }\OperatorTok{-}\StringTok{ }\NormalTok{expected_matrix}
\NormalTok{cont_minus_expected_squared  <-}\StringTok{ }\NormalTok{cont_minus_expected }\OperatorTok{*}\StringTok{ }\NormalTok{cont_minus_expected}
\NormalTok{test_matrix <-}\StringTok{ }\NormalTok{cont_minus_expected_squared }\OperatorTok{/}\StringTok{ }\NormalTok{expected_matrix}
\NormalTok{chi_two_statistic <-}\StringTok{ }\KeywordTok{sum}\NormalTok{(test_matrix)}
\NormalTok{test_chi_two_statistic <-}\StringTok{ }\KeywordTok{chisq.test}\NormalTok{(tab1,}\DataTypeTok{correct=}\OtherTok{FALSE}\NormalTok{)}
\NormalTok{test_chi_two_statistic <-}\StringTok{  }\NormalTok{test_chi_two_statistic}\OperatorTok{$}\NormalTok{statistic}
\NormalTok{test_chi_two_statistic}
\end{Highlighting}
\end{Shaded}

\begin{verbatim}
## X-squared 
##  8.297921
\end{verbatim}

\begin{Shaded}
\begin{Highlighting}[]
\CommentTok{# chi_two_statistic }
\end{Highlighting}
\end{Shaded}

Procent anger antal hundradelar och noteras med \%. Vi kan beräkna hur
många procent \(kn\) är av \(m\) genom att ställa upp
\(100 \times n/m\). Detta fungerar eftersom vår siffersystem är
decimalt. Således blir andelen kvinnor som är för laglig abort i procent
\[100 \times \frac{\text{antal kvinnor för laglig abort}}{\text{totalt antal kvinnor}}\]
och andelen kvinnor som är emot laglig abort i procent
\[100 \times \frac{\text{antal kvinnor mot laglig abort}}{\text{totalt antal kvinnor}}\],
och motsvarande för män. Dessa beräknar ger oss följande tabell med
procent: `

\begin{Shaded}
\begin{Highlighting}[]
\NormalTok{percentages_matrix }\OperatorTok\StringTok{ }\NormalTok{knitr}\OperatorTok{::}\KeywordTok{kable}\NormalTok{()}
\end{Highlighting}
\end{Shaded}

\begin{longtable}[]{@{}lrr@{}}
\toprule
& In Favor & Against\tabularnewline
\midrule
\endhead
Women & 61.80000 & 38.20000\tabularnewline
Men & 53.16667 & 46.83333\tabularnewline
\bottomrule
\end{longtable}

Motsvarande tabell som genereras av den till uppgiften bifogade
testkoden är

\begin{Shaded}
\begin{Highlighting}[]
\KeywordTok{addmargins}\NormalTok{(tab1)}
\end{Highlighting}
\end{Shaded}

\begin{verbatim}
##        opinion
## gender  favor against  Sum
##   women   309     191  500
##   men     319     281  600
##   Sum     628     472 1100
\end{verbatim}

\begin{Shaded}
\begin{Highlighting}[]
\KeywordTok{addmargins}\NormalTok{(}\KeywordTok{prop.table}\NormalTok{(tab1,}\DecValTok{1}\NormalTok{),}\DecValTok{2}\NormalTok{) }
\end{Highlighting}
\end{Shaded}

\begin{verbatim}
##        opinion
## gender      favor   against       Sum
##   women 0.6180000 0.3820000 1.0000000
##   men   0.5316667 0.4683333 1.0000000
\end{verbatim}

Vilket är vår delad på hundra. Så våra uträkningar bör vara ok.

\begin{Shaded}
\begin{Highlighting}[]
\NormalTok{cont_table =}\StringTok{ }\KeywordTok{matrix}\NormalTok{(}
  \KeywordTok{c}\NormalTok{(}\DecValTok{309}\NormalTok{, }\DecValTok{191}\NormalTok{, }\DecValTok{319}\NormalTok{, }\DecValTok{281}\NormalTok{), }\CommentTok{# the data elements }
  \DataTypeTok{nrow=}\DecValTok{2}\NormalTok{,              }\CommentTok{# number of rows }
  \DataTypeTok{ncol=}\DecValTok{2}\NormalTok{,              }\CommentTok{# number of columns }
  \DataTypeTok{byrow =} \OtherTok{TRUE}\NormalTok{)  }\CommentTok{# fill matrix by rows }
\CommentTok{# cont_table}

\NormalTok{women_men =}\StringTok{ }\KeywordTok{rowSums}\NormalTok{(cont_table)}
\CommentTok{# women_men}
\CommentTok{# diag(1 / women_men)}
\NormalTok{favor_against =}\StringTok{ }\KeywordTok{colSums}\NormalTok{(cont_table)}
\CommentTok{# favor_against}
\NormalTok{total =}\StringTok{ }\KeywordTok{sum}\NormalTok{(cont_table)}

\CommentTok{# (percentages_matrix <- diag(1 / women_men) %*% cont_table) * 100}
\CommentTok{# colnames(percentages_matrix) <- c("In Favor", "Against")}
\CommentTok{# rownames(percentages_matrix) <- c("Women", "Men")}
\CommentTok{# percentages_matrix #  %>% knitr::kable()}

\NormalTok{expected_matrix <-}\StringTok{ }\NormalTok{total }\OperatorTok{*}\StringTok{  }\NormalTok{((women_men }\OperatorTok{/}\StringTok{ }\NormalTok{total)) }\OperatorTok\StringTok{  }\KeywordTok{t}\NormalTok{((favor_against }\OperatorTok{/}\StringTok{ }\NormalTok{total))}

\CommentTok{# shift.column(diag((favor_against / total)))}

\CommentTok{# (diag(1 / women_men)  %*% cont_table) }
\CommentTok{# (cont_table %*% diag(1 / favor_against))}

\CommentTok{# total * (cont_table %*% diag(1 / favor_against)) * (diag(1 / women_men)  %*% cont_table)}

\CommentTok{# expected_matrix}
\CommentTok{# cont_table}
\CommentTok{# sum(expected_matrix)}
\NormalTok{obs_divided_by_expected <-}\StringTok{ }\NormalTok{cont_table }\OperatorTok{/}\StringTok{ }\NormalTok{expected_matrix}
\CommentTok{# obs_divided_by_expected}
\CommentTok{# sapply(obs_divided_by_expected, log)}
\CommentTok{# matrix(sapply(obs_divided_by_expected, log), nrow = 2)}
\NormalTok{log_obs_divided_by_expected <-}\StringTok{  }\KeywordTok{matrix}\NormalTok{(}\KeywordTok{sapply}\NormalTok{(obs_divided_by_expected, log), }\DataTypeTok{nrow =} \DecValTok{2}\NormalTok{)}
\CommentTok{# log_obs_divided_by_expected }
\NormalTok{obs_mult_by_log_obs_divided_by_expected <-}\StringTok{ }\NormalTok{cont_table }\OperatorTok{*}\StringTok{ }\NormalTok{log_obs_divided_by_expected}
\CommentTok{# obs_mult_by_log_obs_divided_by_expected}
\CommentTok{# gee_two <- 2 * sum(obs_mult_by_log_obs_divided_by_expected)}
\NormalTok{gee_two <-}\StringTok{ }\DecValTok{2} \OperatorTok{*}\StringTok{ }\NormalTok{(}\DecValTok{309}\OperatorTok{*}\KeywordTok{log}\NormalTok{(}\DecValTok{309}\OperatorTok{/}\NormalTok{((}\DecValTok{628}\OperatorTok{/}\DecValTok{1100}\NormalTok{)}\OperatorTok{*}\NormalTok{(}\DecValTok{500}\OperatorTok{/}\DecValTok{1100}\NormalTok{)}\OperatorTok{*}\DecValTok{1100}\NormalTok{)) }\OperatorTok{+}
\DecValTok{191}\OperatorTok{*}\KeywordTok{log}\NormalTok{(}\DecValTok{191}\OperatorTok{/}\NormalTok{((}\DecValTok{472}\OperatorTok{/}\DecValTok{1100}\NormalTok{)}\OperatorTok{*}\NormalTok{(}\DecValTok{500}\OperatorTok{/}\DecValTok{1100}\NormalTok{)}\OperatorTok{*}\DecValTok{1100}\NormalTok{)) }\OperatorTok{+}
\DecValTok{319}\OperatorTok{*}\KeywordTok{log}\NormalTok{(}\DecValTok{319}\OperatorTok{/}\NormalTok{((}\DecValTok{628}\OperatorTok{/}\DecValTok{1100}\NormalTok{)}\OperatorTok{*}\NormalTok{(}\DecValTok{600}\OperatorTok{/}\DecValTok{1100}\NormalTok{)}\OperatorTok{*}\DecValTok{1100}\NormalTok{)) }\OperatorTok{+}
\DecValTok{281}\OperatorTok{*}\KeywordTok{log}\NormalTok{(}\DecValTok{281}\OperatorTok{/}\NormalTok{((}\DecValTok{472}\OperatorTok{/}\DecValTok{1100}\NormalTok{)}\OperatorTok{*}\NormalTok{(}\DecValTok{600}\OperatorTok{/}\DecValTok{1100}\NormalTok{)}\OperatorTok{*}\DecValTok{1100}\NormalTok{)))}


\CommentTok{# (309/628)*(309/500)*1100}

\CommentTok{# f(x) <- function()}
\end{Highlighting}
\end{Shaded}

\hypertarget{section}{%
\subsection{2.}\label{section}}

Vi vill testa ifall att värdena i kontingnenstabellen skiljer sig
signifikant från de som skulle förväntas ifall att kön och att vara för
eller emot laglig abort vore oberoende? Det vill säga vi vill testa
nollhypotesen att \[
\begin{aligned}
H_{0}: & \text{ den förklaranden variabeln är oberoende av responsvariablen } \Leftrightarrow \\
       &\pi_{ij} = \pi_{i\circ}\pi_{\circ j} \text{ för alla $i$ och $j$}
\end{aligned}
\]

Pearsons \(X^{2}\) statistika kan användas för att testa en hypotetisk
fördelning i en kontingenstabell. Den är \(\chi^{2}(f!!)\) FIXA!
fördelad och beräknas enligt följande

\[
\begin{aligned}
\chi^{2} &= \sum_{i} \sum_{j}  \frac{(\text{observed} - \text{expected})^{2}}{\text{expected}}\\
         &= \sum_{i} \sum_{j}  \frac{(x_{ij} - e_{ij})^{2}}{e_{ij}}
\end{aligned}
\]

där ``expected'', ``\(e_{ij}\)'', är de förväntade värdena i
kontingenstabellen på rad \(i\) och kolumn \(j\) under hypotesen och
``observed'', ``\(x_{ij}\)'', är motsvarande observerade värde i
kontingenstabellen.

Under antagandet att antal som är för respektive emot laglig abort är
oberoende av kön gäller att \[
\pi_{ij} = \pi_{i\circ}\pi_{\circ j}
\] där \(\pi_{ij}\) anger sannoliketen att en observation hamnar på rad
\(i\) och kolumn \(j\), \(\pi_{i\circ}\) anger sannolikheten att en
observation hamnar på rad \(i\) och \(\pi_{\circ j}\) anger
sannolikheten att en observation hamnar på kolumn \(j\). Sannolikheten
\(\pi_{ij}\) kan skattas med hjälp av observationera i kontingens
tabellen på följande vis
\[\widehat{\pi}_{ij} = \frac{n_{ij}}{n_{tot}}\], där \(n_{tot}\) anger
det totala antalet observationer. Vi kan även skatta sannolikheten att
en observation hamnar i rad \(i\) med
\(\widehat{\pi}_{i\circ} = \frac{n_{i+}}{n_{tot}}\), där \(n_{i+}\)
anger summan av alla observationer på rad \(i\) och \(n_{+j}\)
motsvarande för kolumn \(j\). Det ger oss att under nollhypotesen
följande skattningar av de förväntade värdena
\(\widehat{e}_{ij} = n\widehat{\pi}_{i\circ}\widehat{\pi}_{\circ j}\).
Dessa uträkningar ger oss vår chitvå statiska med HUR MÅNGA
FRIHETSGRADER 8.2979214 som vi kan jämföra med en tabell.

Motsvarande \(X^{2}\) statistika som genereras av den till uppgiften
bifogade testkoden är 8.2979214.

\begin{Shaded}
\begin{Highlighting}[]
\KeywordTok{library}\NormalTok{(tidyverse)}
\CommentTok{# install.packages("kableExtra")}
\NormalTok{n11 <-}\StringTok{ }\DecValTok{309}
\NormalTok{n12 <-}\StringTok{ }\DecValTok{191}
\NormalTok{n21 <-}\StringTok{ }\DecValTok{319}
\NormalTok{n22 <-}\StringTok{ }\DecValTok{281}

\NormalTok{cont_table =}\StringTok{ }\KeywordTok{matrix}\NormalTok{(}
  \KeywordTok{c}\NormalTok{(}\DecValTok{309}\NormalTok{, }\DecValTok{191}\NormalTok{, }\DecValTok{319}\NormalTok{, }\DecValTok{281}\NormalTok{), }\CommentTok{# the data elements }
  \DataTypeTok{nrow=}\DecValTok{2}\NormalTok{,              }\CommentTok{# number of rows }
  \DataTypeTok{ncol=}\DecValTok{2}\NormalTok{,              }\CommentTok{# number of columns }
  \DataTypeTok{byrow =} \OtherTok{TRUE}\NormalTok{)  }\CommentTok{# fill matrix by rows }
\CommentTok{# cont_table}

\NormalTok{women_men =}\StringTok{ }\KeywordTok{rowSums}\NormalTok{(cont_table)}
\NormalTok{favor_against =}\StringTok{ }\KeywordTok{colSums}\NormalTok{(cont_table)}
\NormalTok{total =}\StringTok{ }\KeywordTok{sum}\NormalTok{(cont_table)}

\NormalTok{stickprovsoddskvot <-}\StringTok{ }\NormalTok{(n22 }\OperatorTok{*}\StringTok{ }\NormalTok{n11)}\OperatorTok{/}\NormalTok{(n12 }\OperatorTok{*}\StringTok{ }\NormalTok{n21)}


\CommentTok{# mapply(cont_table, f())}
\end{Highlighting}
\end{Shaded}

Vi kan även testa \(H_{0}\) med hjälp av \(G^{2}\) statistikan, som
beräknas som följer

\[
\begin{aligned} 
G^{2} &= 2 \sum_{i} \sum_{j} \text{ observed } \times \log \left( 
\frac{\text{observed}}{\text{expected}} 
\right)\\
&=2 \sum_{i} \sum_{j}  x_{ij} \log 
\left( 
\frac{x_{ij}}{e_{ij}} 
\right)
\end{aligned}
\]

Vid inmatning av värdena från den aktuella kontingenstabellen evalueras
\(G^{2}\) till 8.3223196. Den \(G^{2}\)-statistikan är VAD!? fördelad
med HUR MÅNGA FRIHETSGRADER? frihetsgrader. Sannolikheten att få de
observerade värdena under nollhypotesen är VAD? BEHöVERGöRAS!!!.

\hypertarget{section-1}{%
\subsection{3.}\label{section-1}}

Oddsen att en man är emot laglig abort ges av

\[
\frac{
\text{ sannolikheten att en man är emot laglig abort }
}{
\text{ sannolikheten att en man är för laglig abort }
}.
\]

Det vill säga oddsen anger hur många gånger sannolikare det är att en
man är emot laglig abort än att den är för laglig abort.

Vi vill med hjälp av de observerade värdena estimera den odds-kvot som
anger hur många gånger större oddsen att en man är emot laglig abort,
denoterad \(\Omega_{m}\), än oddsen att en kvinna är det, denoterad
\(\Omega_{k}\). Det vill säga vi vill estimera oddskvoten

\[ \begin{aligned} 
\text{ oddskvot } &= \frac{\Omega_{m}}{\Omega_{k}} \\ 
&= \frac{\text{ odds att en man är emot laglig abort }}{ \text{ odds att en kvinna är emot laglig abort }}\\ 
&= \frac{\text{ (sannolikheten att en man är emot laglig abort)/(sannolikheten att en man är för laglig abort) }}{ \text{ (sannolikheten att en kvinna är emot laglig abort)/(sannolikheten att en kvinna är för laglig abort) }}\\
&= \frac{\pi_{j = 2| i = 2}/(1-\pi_{j = 2| i = 2})}{\pi_{j = 2 | i = 1}/(1-\pi_{j = 2 | i = 1})}
\end{aligned}\] med stickprovsoddskvoten

\[ \begin{aligned} 
\text{ stickprovsoddskvot } &= \frac{\text{ stickprovsodds att en man är emot laglig abort }}{ \text{ stickprovsodds att en kvinna är emot laglig abort }}\\ &=\frac{\widehat{\Omega}_{m}}{\widehat{\Omega}_{k}}\\
&= \frac{\text{ (stickprovssannolikheten att en man är emot laglig abort)/(stickprovssannolikheten att en man är för laglig abort) }}{ \text{ (stickprovssannolikheten att en kvinna är emot laglig abort)/(stickprovssannolikheten att en kvinna är för laglig abort) }}\\
&= \frac{\widehat{\pi}_{j = 2| i = 2}/(1-\widehat{\pi}_{j = 2| i = 2})}{\widehat{\pi}_{j = 1 | i = 2}/(1-\widehat{\pi}_{j = 1 | i = 2})}\\
&= \frac{(n_{22}/n_{2+})/(n_{21}/n_{2+})}{(n_{12}/n_{1+})/(n_{11}/n_{1+})}\\
&= \frac{n_{22}n_{11}}{n_{12}n_{21}}
\end{aligned}\]

Gör vi dessa uträkningar så får vi stickprovsoddskvoten 1.4250849. Med
andra ord är vårt estimat är oddset att en man är emot laglig abort
1.4250849 så högt som oddset att en kvinna är det. Under antagande att
enkätdatan är multinomialfördelad gäller att standardefelet för
oddskvoten kan estimeras med

\[ \widehat{\sigma}(\log (\text{stickprovsoddskvot})) = \sqrt{\left(\frac{1}{n_{11}} + \frac{1}{n_{12}} + \frac{1}{n_{21}} + \frac{1}{n_{12}}\right)}\]

Se ekvation 3.1 på sid 70 i kurslitteraturen. Vidare, gäller per
normaliteten av \(\log \widehat{\theta}\) för stora stickprov, att

\[ \log \widehat{\theta} \pm z_{0.025} \widehat{\sigma}(\log \widehat{\theta}).\]
ger oss ett konfidensintervall med konfidensgrad \(0.05\). Se sid. 71 i
kurslitteraturen, och vi kan exponentiera ändpunkterna av detta
intervall för att få ett konfidensintervall för \[\widehat{\theta}\].
När vi genomför beräkningarna enligt ovan får vi att

\begin{Shaded}
\begin{Highlighting}[]
\NormalTok{se_logstickprovsoddskvot =}\StringTok{ }\KeywordTok{sqrt}\NormalTok{((}\DecValTok{1}\OperatorTok{/}\NormalTok{n11)}\OperatorTok{+}\NormalTok{(}\DecValTok{1}\OperatorTok{/}\NormalTok{n12)}\OperatorTok{+}\NormalTok{(}\DecValTok{1}\OperatorTok{/}\NormalTok{n21)}\OperatorTok{+}\NormalTok{(}\DecValTok{1}\OperatorTok{/}\NormalTok{n22)) }

\KeywordTok{qnorm}\NormalTok{(}\FloatTok{0.975}\NormalTok{)}
\end{Highlighting}
\end{Shaded}

\begin{verbatim}
## [1] 1.959964
\end{verbatim}

\begin{Shaded}
\begin{Highlighting}[]
\NormalTok{hoger_andpunkt <-}\StringTok{ }\KeywordTok{exp}\NormalTok{(}\KeywordTok{log}\NormalTok{(stickprovsoddskvot) }\OperatorTok{+}\StringTok{ }\KeywordTok{qnorm}\NormalTok{(}\FloatTok{0.975}\NormalTok{) }\OperatorTok{*}\StringTok{ }\NormalTok{se_logstickprovsoddskvot)}
\NormalTok{vanster_andpunkt <-}\StringTok{ }\KeywordTok{exp}\NormalTok{(}\KeywordTok{log}\NormalTok{(stickprovsoddskvot) }\OperatorTok{+}\StringTok{ }\KeywordTok{qnorm}\NormalTok{(}\FloatTok{0.025}\NormalTok{) }\OperatorTok{*}\StringTok{ }\NormalTok{se_logstickprovsoddskvot)}
\CommentTok{# stickprovsoddskvot}
\end{Highlighting}
\end{Shaded}

standardfelet \(\widehat{\sigma}(\text{log(stickprovsoddskvot)}) =\)
0.1231477 och att konfidensintervallet för stickprovsoddskvoten blir
(1.1194823,1.8141128).

Motsvarande estimat när man använder den till uppgiften bifogade
testkoden blir:

\begin{Shaded}
\begin{Highlighting}[]
\KeywordTok{library}\NormalTok{(}\StringTok{'epitools'}\NormalTok{) }\CommentTok{# the epitools package must be installed #}
\KeywordTok{oddsratio}\NormalTok{(tab1,}\DataTypeTok{rev=}\StringTok{"neither"}\NormalTok{)}\OperatorTok{$}\NormalTok{measure }\OperatorTok\StringTok{ }\NormalTok{knitr}\OperatorTok{::}\KeywordTok{kable}\NormalTok{()}
\end{Highlighting}
\end{Shaded}

\begin{longtable}[]{@{}lrrr@{}}
\toprule
& estimate & lower & upper\tabularnewline
\midrule
\endhead
women & 1.000000 & NA & NA\tabularnewline
men & 1.424397 & 1.119315 & 1.814751\tabularnewline
\bottomrule
\end{longtable}

vilket är tillräckligt nära våra beräkningar för att vi ska anta att det
beror på avrundingar eller dylikt.

\hypertarget{section-2}{%
\subsection{4.}\label{section-2}}

Vi vill estimera hur många gånger större sannolikheten är att en man är
emot laglig abort än att en kvinna är emot laglig abort. Det vill säga
vi vill estimera riskkvoten, vilken är kvoten mellan sannoliketen att en
man är emot laglig abort och sannolikheten att en kvinna är emot laglig
abort. Denna ges av

\[
\text{ riskkvot } = \frac{\pi_{j = 2|i = 2}}{\pi_{j = 2|i = 1}}
\] där \(i\) är radindex och \(j\) kolumn index i kontingenstabellen.
Denna estimerar vi med

\[
\begin{aligned}
\text{ stickprovsriskkvot } &= \frac{\widehat{\pi}_{j = 2|i = 2}}{\widehat{\pi}_{j = 2|i = 1}}\\
&= \frac{(n_{22}/n_{2+})}{(n_{12}/n_{1+})}
\end{aligned}
\]

\begin{Shaded}
\begin{Highlighting}[]
\NormalTok{stickprovsriskkvot <-}\StringTok{ }\NormalTok{(n22}\OperatorTok{/}\StringTok{ }\NormalTok{women_men[}\DecValTok{2}\NormalTok{])}\OperatorTok{/}\NormalTok{(n12 }\OperatorTok{/}\StringTok{ }\NormalTok{women_men[}\DecValTok{1}\NormalTok{])}
\NormalTok{se_log_stickprovsriskkvot <-}\StringTok{ }\KeywordTok{sqrt}\NormalTok{(((}\DecValTok{1} \OperatorTok{-}\StringTok{ }\NormalTok{(n22}\OperatorTok{/}\StringTok{ }\NormalTok{women_men[}\DecValTok{2}\NormalTok{]))}\OperatorTok{/}\StringTok{ }\NormalTok{women_men[}\DecValTok{2}\NormalTok{]) }\OperatorTok{+}\StringTok{ }\NormalTok{((}\DecValTok{1} \OperatorTok{-}\StringTok{ }\NormalTok{(n12 }\OperatorTok{/}\StringTok{ }\NormalTok{women_men[}\DecValTok{1}\NormalTok{]))}\OperatorTok{/}\StringTok{ }\NormalTok{women_men[}\DecValTok{1}\NormalTok{]))}
\CommentTok{# women_men[2]}
\CommentTok{# n22}
\end{Highlighting}
\end{Shaded}

och när vi räknar ut denna får vi att den blir 1.2260035. Med andra ord
är vårt estimat att män är 1.2260035 så sannolika att vara emot laglig
abort som kvinnor. Den naturliga logaritmen av detta värde blir
0.0460664. Vi räknar med hjälp av formeln 3.5 på sida 71 ut
standradfelet för den logaritmerade stickprovsriskkvoten nedan:

\[
\begin{aligned}
\widehat{\sigma}\left(
\log \left(
\frac{\widehat{\pi}_{j = 2|i = 2}}{\widehat{\pi}_{j = 2|i = 1}}
\right) 
\right) &= \sqrt{
\frac{1 - \widehat{\pi}_{j = 2|i = 2}}{n_{22}} + \frac{1 - \widehat{\pi}_{j = 2|i = 1}}{n_{12}}
}\\
&= \sqrt{
\frac{1 - n_{22}/n_{2+} }{n_{22}} + \frac{1 - n_{12}/n_{1+}}{n_{12}}
}
\end{aligned}
\]

Vi kan sedan skapa konfidensintervallet för stickprovssriskkvoten med
hjälp av formeln för ändamålet på sid 71 i kurslitteraturen

\[
\log \left(
\frac{
\widehat{\pi}_{j = 2|i = 2}}{
\widehat{\pi}_{j = 2|i = 1}}
\right)
\pm z_{0.025}\widehat{\sigma}
\left(
\log \left(
\frac{
\widehat{\pi}_{j = 2|i = 2}}{\widehat{\pi}_{j = 2|i = 1}}
\right)
\right)
\].

Det vill säga

\[
\log \left(
\frac{
n_{22}/ n_{2+} }{
n_{12} / n_{1+}}
\right)
\pm z_{0.025}\sqrt{
\frac{1 - n_{22}/n_{2+} }{n_{22}} + \frac{1 - n_{12}/n_{1+}}{n_{12}}
}
\]. Vilket ger oss konfidensintervallet

\begin{Shaded}
\begin{Highlighting}[]
\NormalTok{vanster_andpunkt2 <-}\StringTok{ }\KeywordTok{exp}\NormalTok{(}
  \KeywordTok{log}\NormalTok{((n22}\OperatorTok{/}\StringTok{ }\NormalTok{women_men[}\DecValTok{2}\NormalTok{])}\OperatorTok{/}\NormalTok{(n12 }\OperatorTok{/}\StringTok{ }\NormalTok{women_men[}\DecValTok{1}\NormalTok{])) }\OperatorTok{+}\StringTok{ }\KeywordTok{qnorm}\NormalTok{(}\FloatTok{0.025}\NormalTok{)}\OperatorTok{*}\KeywordTok{sqrt}\NormalTok{((}\DecValTok{1} \OperatorTok{-}\StringTok{ }\NormalTok{n22}\OperatorTok{/}\NormalTok{women_men[}\DecValTok{2}\NormalTok{])}\OperatorTok{/}\NormalTok{n22 }\OperatorTok{+}\StringTok{ }\NormalTok{(}\DecValTok{1} \OperatorTok{-}\StringTok{ }\NormalTok{n12}\OperatorTok{/}\NormalTok{women_men[}\DecValTok{1}\NormalTok{])}\OperatorTok{/}\NormalTok{n12)}
\NormalTok{)}

\NormalTok{hoger_andpunkt2 <-}\StringTok{ }\KeywordTok{exp}\NormalTok{(}
  \KeywordTok{log}\NormalTok{((n22}\OperatorTok{/}\StringTok{ }\NormalTok{women_men[}\DecValTok{2}\NormalTok{])}\OperatorTok{/}\NormalTok{(n12 }\OperatorTok{/}\StringTok{ }\NormalTok{women_men[}\DecValTok{1}\NormalTok{])) }\OperatorTok{+}\StringTok{ }\KeywordTok{qnorm}\NormalTok{(}\FloatTok{0.975}\NormalTok{)}\OperatorTok{*}\KeywordTok{sqrt}\NormalTok{((}\DecValTok{1} \OperatorTok{-}\StringTok{ }\NormalTok{n22}\OperatorTok{/}\NormalTok{women_men[}\DecValTok{2}\NormalTok{])}\OperatorTok{/}\NormalTok{n22 }\OperatorTok{+}\StringTok{ }\NormalTok{(}\DecValTok{1} \OperatorTok{-}\StringTok{ }\NormalTok{n12}\OperatorTok{/}\NormalTok{women_men[}\DecValTok{1}\NormalTok{])}\OperatorTok{/}\NormalTok{n12)}
\NormalTok{)}
\CommentTok{#   }
\CommentTok{# (n22/ women_men[2])/(n12 / women_men[1]) }

\KeywordTok{library}\NormalTok{(}\StringTok{'epitools'}\NormalTok{) }\CommentTok{# the epitools package must be installed #}
\KeywordTok{oddsratio}\NormalTok{(tab1,}\DataTypeTok{rev=}\StringTok{"neither"}\NormalTok{)}
\end{Highlighting}
\end{Shaded}

\begin{verbatim}
## $data
##        opinion
## gender  favor against Total
##   women   309     191   500
##   men     319     281   600
##   Total   628     472  1100
## 
## $measure
##        odds ratio with 95% C.I.
## gender  estimate    lower    upper
##   women 1.000000       NA       NA
##   men   1.424397 1.119315 1.814751
## 
## $p.value
##        two-sided
## gender   midp.exact fisher.exact  chi.square
##   women          NA           NA          NA
##   men   0.003990219  0.004071121 0.003969048
## 
## $correction
## [1] FALSE
## 
## attr(,"method")
## [1] "median-unbiased estimate & mid-p exact CI"
\end{verbatim}

(1.065465, 1.4107311) med konfidensgrad \(0.05\). Vi kan notera att noll
inte ligger på detta intervall. Således är riskkvoten signifikant skild
från noll.

Motsvarande estimat när man använder den till uppgiften bifogade
testkoden blir:

\begin{Shaded}
\begin{Highlighting}[]
\KeywordTok{library}\NormalTok{(}\StringTok{'epitools'}\NormalTok{) }\CommentTok{# the epitools package must be installed #}
\KeywordTok{riskratio}\NormalTok{(tab1,}\DataTypeTok{rev=}\StringTok{"neither"}\NormalTok{)}\OperatorTok{$}\NormalTok{measure }\OperatorTok\StringTok{ }\NormalTok{knitr}\OperatorTok{::}\KeywordTok{kable}\NormalTok{()}
\end{Highlighting}
\end{Shaded}

\begin{longtable}[]{@{}lrrr@{}}
\toprule
& estimate & lower & upper\tabularnewline
\midrule
\endhead
women & 1.000000 & NA & NA\tabularnewline
men & 1.226004 & 1.065465 & 1.410731\tabularnewline
\bottomrule
\end{longtable}

Vid jämförande av våra resultat med dessa finner vi att det är små
skillnader som förhoppningsvis kan förklaras av avrundningar eller
motsvarande.

\hypertarget{section-3}{%
\section{1.2}\label{section-3}}

\hypertarget{odividerad-kontingenstabellen}{%
\subsection{Odividerad
kontingenstabellen}\label{odividerad-kontingenstabellen}}

\begin{Shaded}
\begin{Highlighting}[]
\NormalTok{m11 =}\StringTok{ }\DecValTok{557}
\NormalTok{m12 =}\StringTok{ }\DecValTok{1835} \OperatorTok{-}\StringTok{ }\DecValTok{557}
\NormalTok{m21 =}\StringTok{ }\DecValTok{1198}
\NormalTok{m22 =}\StringTok{ }\DecValTok{2691} \OperatorTok{-}\StringTok{ }\DecValTok{1198}

\NormalTok{cont_table2 <-}\StringTok{ }\KeywordTok{as.table}\NormalTok{(}\KeywordTok{rbind}\NormalTok{(}\KeywordTok{c}\NormalTok{(m11, m12), }\KeywordTok{c}\NormalTok{(m21, m22)))}
\KeywordTok{dimnames}\NormalTok{(cont_table2) <-}\StringTok{ }\KeywordTok{list}\NormalTok{(}\DataTypeTok{gender =} \KeywordTok{c}\NormalTok{(}\StringTok{"women"}\NormalTok{, }\StringTok{"men"}\NormalTok{), }\DataTypeTok{admission =} \KeywordTok{c}\NormalTok{(}\StringTok{"admitted"}\NormalTok{,}\StringTok{"not_admitted"}\NormalTok{))}
\CommentTok{# cont_table2}
\CommentTok{# cont_table2 <- matrix(}
\CommentTok{#   c(m11, m12, m21, m22), # the data elements }
\CommentTok{#   nrow=2,              # number of rows }
\CommentTok{#   ncol=2,              # number of columns }
\CommentTok{#   byrow = TRUE)  # fill matrix by rows }
\CommentTok{# rownames(cont_table2) <- c('women', 'men')}
\CommentTok{# colnames(cont_table2) <- c('admitted', 'not_admitted')}
\CommentTok{# cont_table2}

\NormalTok{women_men2 <-}\StringTok{ }\KeywordTok{rowSums}\NormalTok{(cont_table2)}
\NormalTok{admitted_not2 <-}\StringTok{ }\KeywordTok{colSums}\NormalTok{(cont_table2)}
\NormalTok{total2 <-}\StringTok{ }\KeywordTok{sum}\NormalTok{(cont_table2)}

\NormalTok{percentage_matrix2 <-}\StringTok{ }\NormalTok{(}\KeywordTok{diag}\NormalTok{(}\DecValTok{1} \OperatorTok{/}\StringTok{ }\NormalTok{women_men2) }\OperatorTok\StringTok{ }\NormalTok{cont_table2) }\OperatorTok{*}\StringTok{ }\DecValTok{100}
\KeywordTok{rownames}\NormalTok{(percentage_matrix2) <-}\StringTok{ }\KeywordTok{c}\NormalTok{(}\StringTok{"women"}\NormalTok{, }\StringTok{"men"}\NormalTok{)}


\CommentTok{#Calculate X2, G2 and p-values}
\NormalTok{pearsonX2 <-}\StringTok{ }\KeywordTok{chisq.test}\NormalTok{(cont_table2,}\DataTypeTok{correct=}\OtherTok{FALSE}\NormalTok{)}
\CommentTok{# pearsonX2}
\NormalTok{pearsonX2_statistic <-}\StringTok{ }\NormalTok{pearsonX2}\OperatorTok{$}\NormalTok{statistic}
\NormalTok{pearsonX2_p <-}\StringTok{ }\NormalTok{pearsonX2}\OperatorTok{$}\NormalTok{p.value}

\KeywordTok{library}\NormalTok{(MASS)     }\CommentTok{# the MASS package must be installed (Rstudio menu bar: Tools - Install packages...)}
\end{Highlighting}
\end{Shaded}

\begin{verbatim}
## 
## Attaching package: 'MASS'
\end{verbatim}

\begin{verbatim}
## The following object is masked from 'package:dplyr':
## 
##     select
\end{verbatim}

\begin{Shaded}
\begin{Highlighting}[]
\CommentTok{# summary(loglm(~gender+admission,cont_table2))%>% knitr::kable()}
\CommentTok{# summary(loglm(~gender+admission,cont_table2))}
\NormalTok{gee2pearsonX2 <-}\StringTok{ }\KeywordTok{summary}\NormalTok{(}\KeywordTok{loglm}\NormalTok{(}\OperatorTok{~}\NormalTok{gender}\OperatorTok{+}\NormalTok{admission,cont_table2))}
\CommentTok{# gee2pearsonX2$tests}

\CommentTok{# gee2_statistic <- gee2pearsonX2$tests}

\CommentTok{#------------------------------------------------------------------------------------------}

\CommentTok{#Calculate risk/odds ratio and confidence interval}
\KeywordTok{library}\NormalTok{(epitools)    }\CommentTok{# the epitools package must be installed #}
\NormalTok{oddsratio2 <-}\StringTok{ }\KeywordTok{oddsratio}\NormalTok{(cont_table2,}\DataTypeTok{rev=}\StringTok{"columns"}\NormalTok{)}
\NormalTok{oddsratio2}
\end{Highlighting}
\end{Shaded}

\begin{verbatim}
## $data
##        admission
## gender  not_admitted admitted Total
##   women         1278      557  1835
##   men           1493     1198  2691
##   Total         2771     1755  4526
## 
## $measure
##        odds ratio with 95% C.I.
## gender  estimate    lower    upper
##   women  1.00000       NA       NA
##   men    1.84063 1.624567 2.087058
## 
## $p.value
##        two-sided
## gender  midp.exact fisher.exact chi.square
##   women         NA           NA         NA
##   men            0 4.835903e-22 7.8136e-22
## 
## $correction
## [1] FALSE
## 
## attr(,"method")
## [1] "median-unbiased estimate & mid-p exact CI"
\end{verbatim}

\begin{Shaded}
\begin{Highlighting}[]
\NormalTok{odds_ratio2statistic <-}\StringTok{ }\NormalTok{oddsratio2}\OperatorTok{$}\NormalTok{measure}

\CommentTok{# odds_ratio2statistic}
\CommentTok{# oddsratio2$measure}

\NormalTok{riskratio2 <-}\StringTok{ }\KeywordTok{riskratio}\NormalTok{(cont_table2, }\DataTypeTok{rev=}\StringTok{"columns"}\NormalTok{)}
\NormalTok{riskratio2}
\end{Highlighting}
\end{Shaded}

\begin{verbatim}
## $data
##        admission
## gender  not_admitted admitted Total
##   women         1278      557  1835
##   men           1493     1198  2691
##   Total         2771     1755  4526
## 
## $measure
##        risk ratio with 95% C.I.
## gender  estimate   lower    upper
##   women 1.000000      NA       NA
##   men   1.466642 1.35235 1.590592
## 
## $p.value
##        two-sided
## gender  midp.exact fisher.exact chi.square
##   women         NA           NA         NA
##   men            0 4.835903e-22 7.8136e-22
## 
## $correction
## [1] FALSE
## 
## attr(,"method")
## [1] "Unconditional MLE & normal approximation (Wald) CI"
\end{verbatim}

I procentmatrisen nedan kan vi se att avrundat till inga decimaler så
blev \(45 \%\) av männen och \(30 \%\) av kvinnorna antagna.

\begin{Shaded}
\begin{Highlighting}[]
\NormalTok{percentage_matrix2 }\OperatorTok\StringTok{ }\NormalTok{knitr}\OperatorTok{::}\KeywordTok{kable}\NormalTok{()}
\end{Highlighting}
\end{Shaded}

\begin{longtable}[]{@{}lrr@{}}
\toprule
& admitted & not\_admitted\tabularnewline
\midrule
\endhead
women & 30.35422 & 69.64578\tabularnewline
men & 44.51877 & 55.48123\tabularnewline
\bottomrule
\end{longtable}

Både likelihood ratio testet av oberoende med \(G^{2}\) statistikan och
Pearsons \(X^{2}\)-test av oberoende låter oss avfärda nollhypotesen att
variablerna är oberoende med de flesta vanligt förekommande
signifikanströsklar.

\begin{Shaded}
\begin{Highlighting}[]
\NormalTok{gee2pearsonX2}\OperatorTok{$}\NormalTok{tests }\OperatorTok\StringTok{ }\NormalTok{knitr}\OperatorTok{::}\KeywordTok{kable}\NormalTok{()}
\end{Highlighting}
\end{Shaded}

\begin{longtable}[]{@{}lrrr@{}}
\toprule
& X\^{}2 & df & P(\textgreater{} X\^{}2)\tabularnewline
\midrule
\endhead
Likelihood Ratio & 93.44941 & 1 & 0\tabularnewline
Pearson & 92.20528 & 1 & 0\tabularnewline
\bottomrule
\end{longtable}

Under nollhypotesen att variablerna är oberoende är populations
oddskvoten
\[\frac{\text{ oddsen att en man blir antagen }}{\text{ oddsen att en kvinna blir antagen }} = 1\].
I tabellen nedan framgår det att oddset att en man blir antagen
estimeras till \(1.84\) gånger så högt som att en kvinna blir det av
stickprovsoddskvoten, med ett konfidensintervall med konfidensgrad
\(.05\) som inte innefattar ett. Det vill säga denna stickprovs kvot är
signifikant ettskild.

\begin{Shaded}
\begin{Highlighting}[]
\NormalTok{oddsratio2}\OperatorTok{$}\NormalTok{measure }\OperatorTok\StringTok{ }\NormalTok{knitr}\OperatorTok{::}\KeywordTok{kable}\NormalTok{()}
\end{Highlighting}
\end{Shaded}

\begin{longtable}[]{@{}lrrr@{}}
\toprule
& estimate & lower & upper\tabularnewline
\midrule
\endhead
women & 1.00000 & NA & NA\tabularnewline
men & 1.84063 & 1.624567 & 2.087058\tabularnewline
\bottomrule
\end{longtable}

Under nollhypotesen att variablerna är oberoende gäller motsvarande
förhållande för riskkvoten:
\[\frac{\text{ sannolikheten att en man blir antagen }}{\text{ sannolikheten att en kvinna blir antagen }} = 1\].
I tabellen nedan estimeras av stickprovsriskkvoten sannolikheten att en
man blir antagen till avrundat \(1.47\) gånger så hög som att en kvinna
blir det, med ett konfidensintervall med konfidensgrad \(.05\) som inte
innefattar ett. Det vill säga denna stickprovskvot är signifikant
ettskild.

\begin{Shaded}
\begin{Highlighting}[]
\NormalTok{riskratio2}\OperatorTok{$}\NormalTok{measure }\OperatorTok\StringTok{ }\NormalTok{knitr}\OperatorTok{::}\KeywordTok{kable}\NormalTok{()}
\end{Highlighting}
\end{Shaded}

\begin{longtable}[]{@{}lrrr@{}}
\toprule
& estimate & lower & upper\tabularnewline
\midrule
\endhead
women & 1.000000 & NA & NA\tabularnewline
men & 1.466642 & 1.35235 & 1.590592\tabularnewline
\bottomrule
\end{longtable}

\hypertarget{kontingenstabellen-delad-med-10}{%
\subsection{Kontingenstabellen delad med
10}\label{kontingenstabellen-delad-med-10}}

\begin{Shaded}
\begin{Highlighting}[]
\NormalTok{m11_tenth =}\StringTok{ }\KeywordTok{round}\NormalTok{(m11}\OperatorTok{/}\DecValTok{10}\NormalTok{)}
\NormalTok{m12_tenth =}\StringTok{ }\KeywordTok{round}\NormalTok{(m12}\OperatorTok{/}\DecValTok{10}\NormalTok{)}
\NormalTok{m21_tenth =}\StringTok{ }\KeywordTok{round}\NormalTok{(m21}\OperatorTok{/}\DecValTok{10}\NormalTok{)}
\NormalTok{m22_tenth =}\StringTok{ }\KeywordTok{round}\NormalTok{(m22}\OperatorTok{/}\DecValTok{10}\NormalTok{)}


\NormalTok{cont_table3 <-}\StringTok{ }\KeywordTok{as.table}\NormalTok{(}\KeywordTok{rbind}\NormalTok{(}\KeywordTok{c}\NormalTok{(m11_tenth, m12_tenth), }\KeywordTok{c}\NormalTok{(m21_tenth, m22_tenth)))}
\KeywordTok{dimnames}\NormalTok{(cont_table3) <-}\StringTok{ }\KeywordTok{list}\NormalTok{(}\DataTypeTok{gender =} \KeywordTok{c}\NormalTok{(}\StringTok{"women"}\NormalTok{, }\StringTok{"men"}\NormalTok{), }\DataTypeTok{admission =} \KeywordTok{c}\NormalTok{(}\StringTok{"admitted"}\NormalTok{,}\StringTok{"not_admitted"}\NormalTok{))}

\CommentTok{# cont_table3}

\CommentTok{# cont_table3 <- matrix(}
\CommentTok{#   c(m11_tenth, m12_tenth, m21_tenth, m22_tenth), # the data elements }
\CommentTok{#   nrow=2,              # number of rows }
\CommentTok{#   ncol=2,              # number of columns }
\CommentTok{#   byrow = TRUE)  # fill matrix by rows }
\CommentTok{# cont_table}

\NormalTok{women_men3 =}\StringTok{ }\KeywordTok{rowSums}\NormalTok{(cont_table3)}
\NormalTok{admitted_not3 =}\StringTok{ }\KeywordTok{colSums}\NormalTok{(cont_table3)}
\NormalTok{total3 =}\StringTok{ }\KeywordTok{sum}\NormalTok{(cont_table3)}
\NormalTok{percentage_matrix3 <-}\StringTok{ }\NormalTok{(}\KeywordTok{diag}\NormalTok{(}\DecValTok{1} \OperatorTok{/}\StringTok{ }\NormalTok{women_men3) }\OperatorTok\StringTok{ }\NormalTok{cont_table3) }\OperatorTok{*}\StringTok{ }\DecValTok{100}
\KeywordTok{rownames}\NormalTok{(percentage_matrix3) <-}\StringTok{ }\KeywordTok{c}\NormalTok{(}\StringTok{"women"}\NormalTok{, }\StringTok{"men"}\NormalTok{)}
\CommentTok{#Calculate X2, G2 and p-values}
\NormalTok{pearsonX2 <-}\StringTok{ }\KeywordTok{chisq.test}\NormalTok{(cont_table2,}\DataTypeTok{correct=}\OtherTok{FALSE}\NormalTok{)}
\CommentTok{# pearsonX2}
\NormalTok{pearsonX2_statistic <-}\StringTok{ }\NormalTok{pearsonX2}\OperatorTok{$}\NormalTok{statistic}
\NormalTok{pearsonX2_p <-}\StringTok{ }\NormalTok{pearsonX2}\OperatorTok{$}\NormalTok{p.value}


\CommentTok{#Calculate X2, G2 and p-values}
\CommentTok{# chisq.test(cont_table3,correct=FALSE)}
\KeywordTok{library}\NormalTok{(MASS)     }\CommentTok{# the MASS package must be installed (Rstudio menu bar: Tools - Install packages...)}
\CommentTok{# loglm(~gender+admission,cont_table2)}

\NormalTok{gee2pearsonX3 <-}\StringTok{ }\KeywordTok{summary}\NormalTok{(}\KeywordTok{loglm}\NormalTok{(}\OperatorTok{~}\NormalTok{gender}\OperatorTok{+}\NormalTok{admission,cont_table3))}

\CommentTok{#------------------------------------------------------------------------------------------}

\CommentTok{#Calculate risk/odds ratio and confidence interval}
\KeywordTok{library}\NormalTok{(epitools)    }\CommentTok{# the epitools package must be installed #}
\NormalTok{oddsratio3 <-}\StringTok{ }\KeywordTok{oddsratio}\NormalTok{(cont_table3,}\DataTypeTok{rev=}\StringTok{"columns"}\NormalTok{)}
\NormalTok{riskratio3 <-}\StringTok{ }\KeywordTok{riskratio}\NormalTok{(cont_table3,}\DataTypeTok{rev=}\StringTok{"columns"}\NormalTok{)}
\end{Highlighting}
\end{Shaded}

I procentmatrisen nedan kan vi se att avrundat till inga decimaler så
blev \(45 \%\) av männen och \(30 \%\) av kvinnorna antagna. Om dessa
siffror påverkats något så är det avrundningen av kvoterna och inte av
divideringen av antalet observationer.

\begin{Shaded}
\begin{Highlighting}[]
\NormalTok{percentage_matrix3 }\OperatorTok\StringTok{ }\NormalTok{knitr}\OperatorTok{::}\KeywordTok{kable}\NormalTok{()}
\end{Highlighting}
\end{Shaded}

\begin{longtable}[]{@{}lrr@{}}
\toprule
& admitted & not\_admitted\tabularnewline
\midrule
\endhead
women & 30.43478 & 69.56522\tabularnewline
men & 44.60967 & 55.39033\tabularnewline
\bottomrule
\end{longtable}

Både likelihood ratio testet av oberoende med \(G^{2}\) statistikan och
Pearsons \(X^{2}\)-test av oberoende låter oss avfärda nollhypotesen att
variablerna är oberoende med säg \(.05\) i signifikanströskel. Dock kan
vi se att statistikorna har ungefär dividerats med tio jämfört med det
odividerade fallet.

\begin{Shaded}
\begin{Highlighting}[]
\NormalTok{gee2pearsonX3}\OperatorTok{$}\NormalTok{tests }\OperatorTok\StringTok{ }\NormalTok{knitr}\OperatorTok{::}\KeywordTok{kable}\NormalTok{()}
\end{Highlighting}
\end{Shaded}

\begin{longtable}[]{@{}lrrr@{}}
\toprule
& X\^{}2 & df & P(\textgreater{} X\^{}2)\tabularnewline
\midrule
\endhead
Likelihood Ratio & 9.364274 & 1 & 0.0022126\tabularnewline
Pearson & 9.240912 & 1 & 0.0023667\tabularnewline
\bottomrule
\end{longtable}

I tabellen nedan framgår det att oddset att en man blir antagen
estimeras till \(1.84\) gånger så högt som att en kvinna blir det av
stickprovsoddskvoten, med ett konfidensintervall med konfidensgrad
\(.05\) som inte innefattar ett. Det vill säga denna stickrprovskvot är
signifikant ettskild. Dock har jämfört med det odividerade fallet
konfidensintervallet blivit ungefär \(1\) enhet längre.

\begin{Shaded}
\begin{Highlighting}[]
\NormalTok{oddsratio3}\OperatorTok{$}\NormalTok{measure }\OperatorTok\StringTok{ }\NormalTok{knitr}\OperatorTok{::}\KeywordTok{kable}\NormalTok{()}
\end{Highlighting}
\end{Shaded}

\begin{longtable}[]{@{}lrrr@{}}
\toprule
& estimate & lower & upper\tabularnewline
\midrule
\endhead
women & 1.000000 & NA & NA\tabularnewline
men & 1.836632 & 1.239821 & 2.741141\tabularnewline
\bottomrule
\end{longtable}

I tabellen nedan framgår det att sannolikheten att en man blir antagen
estimeras till avrundat \(1.47\) gånger så hög som att en kvinna blir
det av stickprovsriskkvoten, med ett konfidensintervall med
konfidensgrad \(.05\) som inte innefattar ett. Det vill säga denna
stickprovskvot är signifikant ettskild. Dock har jämfört med det
odividerade fallet konfidensintervallet blivit ungefär \(0.5\) enhet
längre.

\begin{Shaded}
\begin{Highlighting}[]
\NormalTok{riskratio3}\OperatorTok{$}\NormalTok{measure }\OperatorTok\StringTok{ }\NormalTok{knitr}\OperatorTok{::}\KeywordTok{kable}\NormalTok{()}
\end{Highlighting}
\end{Shaded}

\begin{longtable}[]{@{}lrrr@{}}
\toprule
& estimate & lower & upper\tabularnewline
\midrule
\endhead
women & 1.000000 & NA & NA\tabularnewline
men & 1.465746 & 1.134883 & 1.893069\tabularnewline
\bottomrule
\end{longtable}

\hypertarget{kontingenstabellen-delad-med-100}{%
\subsection{Kontingenstabellen delad med
100}\label{kontingenstabellen-delad-med-100}}

\begin{Shaded}
\begin{Highlighting}[]
\NormalTok{m11_hundreth =}\StringTok{ }\KeywordTok{round}\NormalTok{(m11}\OperatorTok{/}\DecValTok{100}\NormalTok{)}
\NormalTok{m12_hundreth =}\StringTok{ }\KeywordTok{round}\NormalTok{(m12}\OperatorTok{/}\DecValTok{100}\NormalTok{)}
\NormalTok{m21_hundreth =}\StringTok{ }\KeywordTok{round}\NormalTok{(m21}\OperatorTok{/}\DecValTok{100}\NormalTok{)}
\NormalTok{m22_hundreth =}\StringTok{ }\KeywordTok{round}\NormalTok{(m22}\OperatorTok{/}\DecValTok{100}\NormalTok{)}

\NormalTok{cont_table4 <-}\StringTok{ }\KeywordTok{as.table}\NormalTok{(}\KeywordTok{rbind}\NormalTok{(}\KeywordTok{c}\NormalTok{(m11_hundreth, m12_hundreth), }\KeywordTok{c}\NormalTok{(m21_hundreth, m22_hundreth)))}
\KeywordTok{dimnames}\NormalTok{(cont_table4) <-}\StringTok{ }\KeywordTok{list}\NormalTok{(}\DataTypeTok{gender =} \KeywordTok{c}\NormalTok{(}\StringTok{"women"}\NormalTok{, }\StringTok{"men"}\NormalTok{), }\DataTypeTok{admission =} \KeywordTok{c}\NormalTok{(}\StringTok{"admitted"}\NormalTok{,}\StringTok{"not_admitted"}\NormalTok{))}


\NormalTok{cont_table4}
\end{Highlighting}
\end{Shaded}

\begin{verbatim}
##        admission
## gender  admitted not_admitted
##   women        6           13
##   men         12           15
\end{verbatim}

\begin{Shaded}
\begin{Highlighting}[]
\CommentTok{# cont_table4 <- matrix(}
\CommentTok{#   c(m11_hundreth, m12_hundreth, m21_hundreth, m22_hundreth), # the data elements }
\CommentTok{#   nrow=2,              # number of rows }
\CommentTok{#   ncol=2,              # number of columns }
\CommentTok{#   byrow = TRUE)  # fill matrix by rows }
\CommentTok{# cont_table}

\NormalTok{women_men4 =}\StringTok{ }\KeywordTok{rowSums}\NormalTok{(cont_table4)}
\NormalTok{admitted_not4 =}\StringTok{ }\KeywordTok{colSums}\NormalTok{(cont_table4)}
\NormalTok{total4 =}\StringTok{ }\KeywordTok{sum}\NormalTok{(cont_table4)}
\NormalTok{percentage_matrix4 <-}\StringTok{ }\NormalTok{(}\KeywordTok{diag}\NormalTok{(}\DecValTok{1} \OperatorTok{/}\StringTok{ }\NormalTok{women_men4) }\OperatorTok\StringTok{ }\NormalTok{cont_table4) }\OperatorTok{*}\StringTok{ }\DecValTok{100}
\KeywordTok{rownames}\NormalTok{(percentage_matrix4) <-}\StringTok{ }\KeywordTok{c}\NormalTok{(}\StringTok{"women"}\NormalTok{, }\StringTok{"men"}\NormalTok{)}

\CommentTok{#Calculate X2, G2 and p-values}
\CommentTok{# chisq.test(cont_table4,correct=FALSE)}
\KeywordTok{library}\NormalTok{(MASS)     }\CommentTok{# the MASS package must be installed (Rstudio menu bar: Tools - Install packages...)}
\CommentTok{# loglm(~gender+admission,cont_table4)}

\CommentTok{#------------------------------------------------------------------------------------------}

\CommentTok{#Calculate risk/odds ratio and confidence interval}
\KeywordTok{library}\NormalTok{(epitools)    }\CommentTok{# the epitools package must be installed #}
\NormalTok{oddsratio4 <-}\StringTok{ }\KeywordTok{oddsratio}\NormalTok{(cont_table4,}\DataTypeTok{rev=}\StringTok{"columns"}\NormalTok{)}
\NormalTok{riskratio4 <-}\StringTok{ }\KeywordTok{riskratio}\NormalTok{(cont_table4,}\DataTypeTok{rev=}\StringTok{"columns"}\NormalTok{)  }
\NormalTok{gee2pearsonX4 <-}\StringTok{ }\KeywordTok{summary}\NormalTok{(}\KeywordTok{loglm}\NormalTok{(}\OperatorTok{~}\NormalTok{gender}\OperatorTok{+}\NormalTok{admission,cont_table4))}
\end{Highlighting}
\end{Shaded}

I procentmatrisen nedan kan vi se att avrundat till inga decimaler så
blev \(44 \%\) av männen och \(31 \%\) av kvinnorna antagna.
Skillnaderna här mot de andra två fallen beror enbart på avrudningen.

\begin{Shaded}
\begin{Highlighting}[]
\NormalTok{percentage_matrix4 }\OperatorTok\StringTok{ }\NormalTok{knitr}\OperatorTok{::}\KeywordTok{kable}\NormalTok{()}
\end{Highlighting}
\end{Shaded}

\begin{longtable}[]{@{}lrr@{}}
\toprule
& admitted & not\_admitted\tabularnewline
\midrule
\endhead
women & 31.57895 & 68.42105\tabularnewline
men & 44.44444 & 55.55556\tabularnewline
\bottomrule
\end{longtable}

Både likelihood ratio testet av oberoende med \(G^{2}\) statistikan och
Pearsons \(X^{2}\)-test av oberoende låter oss inte längre avfärda
nollhypotesen att variablerna är oberoende med någon rimlig
signifikanströskel. De har krympt med samma magnitud som antalet
observationer.

\begin{Shaded}
\begin{Highlighting}[]
\NormalTok{gee2pearsonX4}\OperatorTok{$}\NormalTok{tests }\OperatorTok\StringTok{ }\NormalTok{knitr}\OperatorTok{::}\KeywordTok{kable}\NormalTok{()}
\end{Highlighting}
\end{Shaded}

\begin{longtable}[]{@{}lrrr@{}}
\toprule
& X\^{}2 & df & P(\textgreater{} X\^{}2)\tabularnewline
\midrule
\endhead
Likelihood Ratio & 0.7833632 & 1 & 0.3761145\tabularnewline
Pearson & 0.7749930 & 1 & 0.3786768\tabularnewline
\bottomrule
\end{longtable}

Nu ser vi även i tabellerna nedan oddskvots- och riskkvotsestimaten ut
att ha påverkats av minskningen av antalet observationer, men det måste
vara avrundningen som spökar eftersom proportionera mellan de olika
cellerna i kontingenstabellen inte rimligen kan påverkas av
divideringen. Dock har konfidensintervallen för estimaten växt, vilket
är väntat, och de omfattar nu ett, så inte heller dessa kan sägas vara
signifikanta.

\begin{Shaded}
\begin{Highlighting}[]
\NormalTok{oddsratio4}\OperatorTok{$}\NormalTok{measure }\OperatorTok\StringTok{ }\NormalTok{knitr}\OperatorTok{::}\KeywordTok{kable}\NormalTok{()}
\end{Highlighting}
\end{Shaded}

\begin{longtable}[]{@{}lrrr@{}}
\toprule
& estimate & lower & upper\tabularnewline
\midrule
\endhead
women & 1.000000 & NA & NA\tabularnewline
men & 1.700141 & 0.4972715 & 6.243172\tabularnewline
\bottomrule
\end{longtable}

\begin{Shaded}
\begin{Highlighting}[]
\NormalTok{riskratio4}\OperatorTok{$}\NormalTok{measure }\OperatorTok\StringTok{ }\NormalTok{knitr}\OperatorTok{::}\KeywordTok{kable}\NormalTok{()}
\end{Highlighting}
\end{Shaded}

\begin{longtable}[]{@{}lrrr@{}}
\toprule
& estimate & lower & upper\tabularnewline
\midrule
\endhead
women & 1.000000 & NA & NA\tabularnewline
men & 1.407407 & 0.6420765 & 3.084984\tabularnewline
\bottomrule
\end{longtable}

\hypertarget{section-4}{%
\subsection{2.3}\label{section-4}}

\begin{Shaded}
\begin{Highlighting}[]
\NormalTok{populationsoddskvot <-}\StringTok{ }\NormalTok{((}\DecValTok{1}\OperatorTok{/}\DecValTok{3}\NormalTok{)}\OperatorTok{/}\NormalTok{(}\DecValTok{1} \OperatorTok{-}\StringTok{ }\DecValTok{1}\OperatorTok{/}\DecValTok{3}\NormalTok{))}\OperatorTok{/}\NormalTok{((}\FloatTok{0.5}\NormalTok{)}\OperatorTok{/}\NormalTok{(}\DecValTok{1} \OperatorTok{-}\StringTok{ }\FloatTok{0.5}\NormalTok{))}
\end{Highlighting}
\end{Shaded}

Vi såg i föregående deluppgift att det räcker med drygt \(10\)
observationer per ruta för att få en signifikanta oddskvoter. Vi räknar
varje gång en balanserad tärning rullar ett jämt resultat som en
framgång och alla andra resultat som ett misslyckande. Denna sanna
oddskvoten för att lyckas mellan en balanserad tresidig tärning och en
balansread sexsidiga tärningen är 0.5. När vi rullar en sexsidig tärning
tjugo gånger och niosidig tärning tjugo gånger får vi

\begin{Shaded}
\begin{Highlighting}[]
\NormalTok{sex_sidor_udda =}\StringTok{ }\DecValTok{10}
\NormalTok{sex_sidor_jamnt =}\StringTok{ }\DecValTok{10}
\NormalTok{tre_sidor_udda =}\StringTok{ }\DecValTok{10}
\NormalTok{tre_sidor_jamnt =}\StringTok{ }\DecValTok{10}

\NormalTok{tarningstabell <-}\StringTok{ }\KeywordTok{as.table}\NormalTok{(}\KeywordTok{rbind}\NormalTok{(}\KeywordTok{c}\NormalTok{(sex_sidor_udda, sex_sidor_jamnt), }\KeywordTok{c}\NormalTok{(tre_sidor_udda, tre_sidor_jamnt)))}

\KeywordTok{dimnames}\NormalTok{(tarningstabell) <-}\StringTok{ }\KeywordTok{list}\NormalTok{(}\DataTypeTok{tarning =} \KeywordTok{c}\NormalTok{(}\StringTok{"sexsidig"}\NormalTok{, }\StringTok{"tresidig"}\NormalTok{), }\DataTypeTok{kast =} \KeywordTok{c}\NormalTok{(}\StringTok{"udda"}\NormalTok{,}\StringTok{"jämnt"}\NormalTok{))}


\NormalTok{tarningsoddskvot <-}\StringTok{ }\KeywordTok{oddsratio}\NormalTok{(tarningstabell)}

\NormalTok{tarningsoddskvot }
\end{Highlighting}
\end{Shaded}

\begin{verbatim}
## $data
##           kast
## tarning    udda jämnt Total
##   sexsidig   10    10    20
##   tresidig   10    10    20
##   Total      20    20    40
## 
## $measure
##           odds ratio with 95% C.I.
## tarning    estimate     lower    upper
##   sexsidig        1        NA       NA
##   tresidig        1 0.2807039 3.562472
## 
## $p.value
##           two-sided
## tarning    midp.exact fisher.exact chi.square
##   sexsidig         NA           NA         NA
##   tresidig          1            1          1
## 
## $correction
## [1] FALSE
## 
## attr(,"method")
## [1] "median-unbiased estimate & mid-p exact CI"
\end{verbatim}

Vi noterar att observerade oddskvoten mellan dessa två tärningar är
\(1 \in (0.99, 1.01)\) men att konfidensintervallet är brett. Jag skulle
inte rapportera detta som stöd för nollhypotesen att variablerna är
oberoende. Om konfidensintervallet var mycket skulle tightare skulle jag
ändå inte ha rapporterat som stöd för nollhypotesen eftersom
populationsoddskvoten mellan dessa två tärningar är välkänd och starkt
teoretiskt förankrad. Undantag under sådana omständigheter kräver
extrema bevis och vi skulle anta att experimentet inte gått rätt till
eller att det skett något extremt osannolikt snarare än att slumpens
lagar ändrat på sig.

\hypertarget{uppgift-3}{%
\section{Uppgift 3}\label{uppgift-3}}

Eftersom moderns alkoholkonsumtion är förklarande variabel och
födelsevikt är responsvariabel och studien inte är prospektiv kommer kan
vi inte estimera \(\pi_{j = 1| i =1}\), sannolikheten att en observation
hamnar i kolumn ett givet att den är på rad \(1\), direkt med
\(\widehat{\pi}_{j = 1| i =1} = n_{11}/n_{1+}\) direkt, efftersom våra
sannolikhetsestimat vore deskriptiv statistisk och inte genuina
sannolikheter, eftersom utfallen i responsvariablarna redan är kända.
Utan vi måste betrakta responsen, bebisarnas vikt, som givna och använda
att vi känner till att sannolikheten att en bebis är för tidigt född med
låg vikt är \(7\) procent. Tack och lov, kan vi använda följande
samband:

\[
\begin{aligned}
  P(A|B)P(A) &= P(A \cap B) = P(B | A)P(A) &\Leftrightarrow\\
      P(A|B) &= \frac{P(B|A)P(B)}{P(A)} &\Leftrightarrow \\
\end{aligned}
\] vilket i det här fallet är

\[
\begin{aligned}
P(\text{Födelsevikten är låg givet att mamman konsumerar måttligt till mycket alkohol})  
\\= \frac{P(\text{Mamman konsumerar måttligt alkohol givet att bebisen har låg vikt)}P(\text{Bebisen har låg vikt})}{P(\text{Mamman konsumerar måttligt till mycket alkohol})}\\ = \frac{\pi_{i = 1  | j = 1}\times0.07}{\pi_{1\circ}} 
\end{aligned}
\]

där \(\pi_{i = 1| j = 1}\) anger sannolikheten att observationen hamnar
på rad \(i\) givet att den är i kolumn \(j\), dvs värdet på förklarande
variabeln givet responsvariabeln, och \(\pi_{1\circ}\) anger
sannolikheten att en observation hamna på rad \(1\).

Vi kan estimera \[
P(\text{Mamman konsumerar måttligt alkohol givet att bebisen har låg vikt})
\] med \(n_{11}/n_{+1} = 0.1\). Vi kan använda ``law total
probability'', som i det här fallet blir på formen
\[P(A) = P(A|B)P(B) + P(A|B^{c})P(B^{c})\] för att estimera

\[
P(\text{Mamman konsumerar måttligt till mycket alkohol})\\
\] med
\[(n_{11}/n_{+1})0.07+ (n_{12}/n_{+2})0.93 \\= 0.1\times0.07+ (2/30)\times0.93 \\= 0.069\]

\hypertarget{section-5}{%
\subsection{3.2}\label{section-5}}

När studien är prosepktiv kan beräkna
\(\widehat{\pi}_{j = 1| i = 1} = n_{11}/n_{1+}\) direkt, vilket blir
\(10/30 = 1/3\).

\hypertarget{section-6}{%
\subsection{3.3}\label{section-6}}

Sannolikheten blir olika på grund av att stickprovet bara är ett av
många möjliga från den population där
\(P(\text{Barnet är för tidigt fött med låg vikt})=0.07\). Det vill säga
stickprovsestimatet för denna sannolikhet är inte \(0.07\), i detta
fall, så likhet råder inte.


\end{document}
